\documentclass[sigconf]{acmart}

\usepackage{hyperref}

\usepackage{endfloat}
\renewcommand{\efloatseparator}{\mbox{}} % no new page between figures

\usepackage{booktabs} % For formal tables

\settopmatter{printacmref=false} % Removes citation information below abstract
\renewcommand\footnotetextcopyrightpermission[1]{} % removes footnote with conference information in first column
\pagestyle{plain} % removes running headers

\begin{document}
\title{Big Data in Social Media}


\author{Shiqi Shen}
\affiliation{%
  \institution{Indiana University Bloomington}
  \streetaddress{1575 S Ira St}
  \city{Bloomington} 
  \state{Indiana} 
  \postcode{47401}
}
\email{shiqshen@indiana.edu}


\begin{abstract}

    Big data refers to data that is generated from various social media platforms. The impact of this data on marketing cannot be overlooked. Many marketers have used analytics to create adverts that are tailored to the specific needs of the users. When a user comments or likes a product on twitter and Facebook, the same information is captured and used to analyze the customer behavior. Facebook uses Hadoop technology to help it di deep into the customers behavior to draw important insights which can then be used to suggest products for users based on their browsing behavior. This paper will evaluate big data and social media, with specific focus on how social media has been employed to improve marketing.
    
\end{abstract}

\keywords{i523, hid109, Big data, Social media, Facebook, Platform, advertising, marketing}

\maketitle

\section{INTRODUCTION}

The notion of big data has been in existence for several years now. Several organizations are now aware of the fact that if they can capture the streams of data flowing into their businesses, then they can employ the power of analytics to get good value from the said data. Basic analytics was used even in the early days even before the powerful analytical tools came into existence. The importance of big data in businesses cannot be overlooked. Businesses have been able to draw the insights from the big data to discover new opportunities. The approach has led to good business decisions, efficient operations, satisfied customers and increased profits \cite{Singh2016}. The tremendous upsurge in the volumes of data can be overwhelming for businesses and individuals who may fail to make any sense from it. However, with the shift in trend, an organization can adjust its strategy to help it compete favorably. Perhaps the most important concept to consider is the role of big data plays in social media marketing techniques. Social media sites including Facebook, Twitter, and Instagram are actually fundamental components of big data and form one of the most important sources of big data. The uninterrupted flow of content from various social media sites has enabled data analytics of the previous years to grow into big data.

\section{SOCIAL MEDIA IS SIGNIFICANT FOR COMPANIES AND INDIVIVUALS}

Although big data is said to come from several different sources, the largest proportion of it is said to originate from unstructured sources. As it can be imagined, social media makes up the largest source of unstructured content for big data. All the activities that users perform on social media such as views, retweets, comments, favorites, likes, and all activities undertaken by social media users to interact on the social media sites can be gathered and explored by interested individuals. \\
In the current digital world, social media plays a vital role in any company. Having a presence on various social media platforms such as Instagram, Facebook, and Twitter is imperative since it enables individuals to interact with an organization on an ostensibly personal level which helps businesses across several domains. Besides, it is imperative for the average consumer as well. For instance, Facebook alone has over two billion users on every month, which is roughly twenty-six percent of the whole population of the world \cite{Geer2017}. It is hence imperative to consider the fact that big data from the social media websites can reach any company in several number of forms. Social media interactions have continued to play a big role and will continue to play a big role in business decisions. For example, some insurance companies have declined to offer life insurance policy to individuals based on their posts on these platforms. If you post on social media every weekend about how you are drinking, insurance companies would be reluctant to offer you life insurance policy as this is a risk to them. \\
Organizations will discover new strategies for making sense of big data. At the moment, the concept of big data is still much evolving. Despite the fact that businesses have found some ways to interact and use this data, this is just the start. For example, a marketing company whose interest is promoting a new product could employ machine learning algorithm that enables it to gather data from individuals who meet certain attributes \cite{Geer2017}. Consequently, by employing artificial intelligence technology, they will be capable of drawing insights from millions of users to create campaigns with levels of precision and focus, a very focused reach, and an excellent opportunity for finding the target audience and satisfying its preferences. 

<<<<<<< HEAD
\section{BIG DATA IN SOCIAL MEDIA ADVERTISING}
=======
The \textit{proceedings} are the \cite{VanGundy09}
>>>>>>> 24c757aceaff2a52e53dedcd8f5ec92599ae9e5c

Fundamentally, advertising revolves around communication since it is aimed at sensitizing consumers regarding products and services that an organization is floating for sale. However, different consumers will always want to hear varied messages, which is the most imperative fact as new clients are recruited into the internet bandwagon because of the growing popularity of the mobile. Big data has the capacity to customize these messages, project what consumers would like to hear, and establish new perceptions on what customers like or prefer \cite{Hensel&Deis2010}. The above steps are all revolutionary and are expected to have a significant impact on how marketers in various organizations will use advertising.  \\
Some occurrences which several people do not view as advertising are an interaction between big data and marketing. An obvious example is Netflix \cite{Eastwood2017}. Although the company does not have a concrete advertisement, it employees a lot of algorithms to recommend various movies and shows to the customers. The approach saves the organization a lot of money by reducing the rate of customer exit and ensuring that the right shows are marketed to the right individuals. The company’s strategies to target consumers with shows specifically tailored for them is not unusual, with other firms such as Amazon, YouTube, etc. also do the same most often \cite{Eastwood2017}. To stay up to date, the algorithms need a constant flow of data. With the growth of the internet, users leave huge volumes of data not only on social media platforms but also everywhere they go in the form of digital footprint.  This provides advertisers with new avenues to tailor their messages to meet the customer's demands specifically. \\
The digital footprints left by online advertisers provides new insights to marketers into what a consumer really needs, and this may be more accurate than what the customer actually says on social media. However, marketers are worried about how to safeguard the privacy and security of their consumers. Companies that are careless in handling data collected from consumers will ignite a backlash which may greatly impact consumer’s business. Targeted advertising has been in existence for quite a while \cite{Eastwood2017}. As more data is collected by advertisers, the targeted advertising approach is expected to be more personalized and more effective. Organizations will strive not just to gather as much data as they can, but also to gather information which typically represents the individual consumers to market to their personalized tastes. 

\section{ANALYZING LINKS}

Big data collected from social media can lead to the discovery of new information regarding each individual customer, although the message must be customized to appeal to the specific customer. However, with the new insights, marketers can enhance how advertising is approached as they create new strategies. The growth in content marketing is usually perceived as a primary beneficiary of big data, although the concept of content marketing could be older than the internet itself. \\
Big data enables digital marketers to target users effectively with more personalized advertisements that they prefer to see. Facebook and Google, etc. are among the biggest players in the domain if digital advertising. They have discovered excellent ways to create and deliver more appealing advertisements in ways that do not intrude on the rights and preferences of the consumers \cite{Mangold&Faulds2009}. Most advertisements are featuring services and goods that consumers would like most to enhance their lives. All these advertisements are reliant on huge loads of personal data that users usually provide about what they are up to, sharing, liking, as well as where they are going.\\
Experts contend that it is possible to accurately make predictions on an array of individual attributes that are more sensitive merely through an analysis of an individual’s Facebook or Twitter likes \cite{Nate2014}. For example, the likes on these social media websites are critical in predicting the one’s religion, sexual orientation, emotional stability, life satisfaction, age, relationship status, and many other attributes. For instance, Facebook successfully linked political activity with user commitment when created a sticker enabling users to declare on their profiles that they had voted. The initiative was conducted during 2010 id term polls and was much effective. Individuals who saw the feature had high chances of voting and actively engage in a conversation about the same after seeing their friends and peers participate in the activity. During the 2016 polls, Facebook escalated their role into the voting process by providing users with constant reminders and directions about their polling stations.

\section{USER RATING AND POP UP ADS}
Depending on the user preferences and the content that they often access on social media, pop-up advertisements can be created to target them every time they are online specifically. For example, an ad can be created on the Facebook Messenger app to open inside the particular app every time the user hits the CTA button. When clicked, such adverts would redirect the user to a page where they would be required to answer some question, claim a reward or send some feedback regarding a product or service. Before creating such ads, it is imperative to establish a custom audience of the individuals who would be targeted with the particular pop-up ads. For instance, individuals who have previously liked the company’s products on the Facebook page or other social media sites can be included on the list of targets. Another strategy that can be employed to rate users is tracking of cookies. The activities of the users are tracked across the internet using tracking cookies. For example, if a user is logged into one of the social media sites and concurrently browses sites, then the sites they are visiting can be tracked.

\section{CONCLUSION}
Big data and social media have revolutionized advertising. Many organizations are nowadays harnessing the power of social media and analytics to create marketing campaigns that are targeted specifically to consumer needs. Users interact in various ways in the social media platforms, For example, on Facebook and Twitter, users can comment on a status, like, retweet, or share. These activities generate huge loads of data about a user’s behavior. The data can be analyzed and used to create marketing campaigns that specifically target the users. The approach has proved to be effective as most organizations have experienced increased customers and sales.

\begin{acks}

  The authors would like to thank Dr. Gregor von Laszewski for his support to write this paper as well as TAs' helpful suggestions on this paper. 

\end{acks}

\bibliographystyle{ACM-Reference-Format}
\bibliography{report} 

\end{document}
