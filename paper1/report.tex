\documentclass[sigconf]{acmart}

\usepackage{graphicx}
\usepackage{hyperref}
\usepackage{todonotes}

\usepackage{endfloat}
\renewcommand{\efloatseparator}{\mbox{}} % no new page between figures

\usepackage{booktabs} % For formal tables

\settopmatter{printacmref=false} % Removes citation information below abstract
\renewcommand\footnotetextcopyrightpermission[1]{} % removes footnote with conference information in first column
\pagestyle{plain} % removes running headers

\newcommand{\TODO}[1]{\todo[inline]{#1}}

\begin{document}
\title{Big Data in Social Media}


\author{Shiqi Shen}
\affiliation{%
  \institution{Indiana University Bloomington}
  \streetaddress{1575 S Ira St}
  \city{Bloomington} 
  \state{Indiana} 
  \postcode{47401}
}
\email{shiqshen@indiana.edu}


\begin{abstract}

Big data refers to data that is generated from various sources including social media platforms. The impact of this data on marketing cannot be overlooked as many marketers have used big data analytics to create adverts that are tailored to the specific needs of users. This is possible because whenever a user comments or likes a product on twitter and Facebook, the same information is captured and used to analyze the customer’s behavior. Facebook uses Hadoop technology to help it dig deep into the customer’s behavior and draw important insights which can be used to suggest products for users based on their browsing patterns. This paper will evaluate big data and social media with specific focus on how big data, through social media, has been employed to improve marketing.
    
\end{abstract}

\keywords{i423, hid109, Big data, Social media, Facebook, Harnessed,  Platform, advertising, marketing}

\maketitle

\section{INTRODUCTION}

Big data has been in existence for a number of years now. During this time several organizations have become aware of the fact that if they capture the streams of data flowing into their businesses, they can employ the power of big data analytics to get good value from it. Before this, basic analytics was being used in the early days as powerful analytical tools had not yet come into existence. The importance of big data in businesses cannot be overlooked as businesses have been able to draw valuable insights from big data analytics which has played a big part in helping them discover new opportunities. This approach has led to good business decisions, efficient operations, satisfied customers and increased profits \cite{Singh2016}. In many cases the tremendous upsurge in the volumes of data can sometimes be overwhelming for businesses and individuals who may not know how to make sense of it all. However, with the shift in trend, an organization can adjust its strategy to help it compete favorably by using big data analytics. Subsequently, the most important concept to consider is the role that big data plays in social media marketing techniques. Social media sites, including Facebook, Twitter, and Instagram, are fundamental components of big data and form one of the most important sources of big data. The uninterrupted flow of content from various social media sites has enabled data collected from previous years to grow into big data.

\section{SOCIAL MEDIA IS SIGNIFICANT FOR COMPANIES AND INDIVIVUALS}

Although big data is said to come from several different sources, the largest proportion of it is said to originate from unstructured sources. As it can be imagined, social media makes up the largest source of unstructured content for big data. All the activities that users perform on social media such as views, retweets, comments, favorites, likes, etc. can be gathered and explored by interested individuals. \\
In the current digital world, social media plays a vital role in many companies. Having a presence on various social media platforms such as Instagram, Facebook, and Twitter is imperative since it enables individuals to interact with an organization on an ostensibly personal level and at the same time helps businesses across several domains get in touch with their customers. Currently, Facebook alone has over two billion users on their platform; this is roughly twenty-six percent of the world population \cite{Geer2017}. It is therefore important to consider the fact that big data, from the social media platforms, can reach any people in different forms. Besides that, social media interactions have continued to play a big role and will continue to play a big role in business decisions. For example, some insurance companies have declined to offer life insurance policies to individuals solely based on their social media posts. If you frequently post, on any of these platforms, about how you are drinking or going to drink, insurance companies would be reluctant to offer you a life insurance policy as this is a risk to them. \\
It will not be long before organizations discover new and better strategies for making sense of big data. But, at the moment, the concept of big data is still new and rapidly evolving. Nevertheless, some businesses have found ways of interacting and using this data, which is just but the beginning, but still a good way to begin. To elaborate, a marketing company whose interest is promoting a new product could employ machine learning algorithms that enable it to gather data from individuals who meet certain attributes \cite{Geer2017}. Consequently, by employing artificial intelligence technology, they will also be capable of drawing insights from millions of users and create campaigns. This will increase their levels of precision and focus, a technique usually referred to as targeted marketing, and present an excellent opportunity for finding the perfect audience and satisfy its preferences. 

\section{BIG DATA IN SOCIAL MEDIA ADVERTISING}

Fundamentally, advertising revolves around communication since it is all about sensitizing consumers on products and services that an organization is selling. However, different consumers will always want to hear varied messages, which is a vital fact to consider when new clients are being recruited into the internet bandwagon due to the growing popularity of smart phones. Big data has the capacity to customize these messages, project what consumers would like to hear, and establish new perceptions on what customers like or prefer \cite{HenselandDeis2010}. The above steps are all revolutionary and are expected to have a significant impact on how marketers in various organizations advertise.   \\
Furthermore, there are some occurrences which several people do not view as advertising but are still interactions between big data and marketing like product recommendation. An obvious example is Netflix \cite{Eastwood2017}. Although the company does not have a concrete advertisement plan, it employees a lot of algorithms to recommend various movies and shows to its customers. The approach saves the organization a lot of money by reducing the rate of customer exit and ensures that the right shows are marketed to the right individuals. The company’s strategy is to target consumers with shows specifically tailored for them. Apart from them, other firms such as Amazon, YouTube, etc. also do the same by using product recommendation to target their customers  \cite{Eastwood2017}. In order to stay up to date, the algorithms need constant flow of data to help it work more efficiently. With the growth of the internet, users leave huge volumes of data not only on social media platforms but also on other places they visit online in the form of a digital footprint.  This provides advertisers with new avenues to tailor their messages to meet their customer demands.  \\
The digital footprints left by online advertisers provides new insights to marketers on what a consumer really needs, and this sometimes may be more accurate than what the customer actually says on social media. However, marketers are worried about how to safeguard the privacy and security of their consumers and therefore companies that are careless in handling data collected from consumers usually ignite a backlash which greatly impact their business. Even though targeted advertising has been in existence for quite a while \cite{Eastwood2017} the more the data that is collected by advertisers, the more personalized and effective marketing is expected to be. Organizations will strive not just to gather as much data as they can, but also to gather information which typically represents the individual consumer’s needs in order to enable them to market to their personalized tastes.

\section{ANALYZING LINKS}

Big data collected from social media can lead to the discovery of new information regarding each individual customer that can help in creating a customized appeal to that specific customer. However, with the new insights, marketers can enhance how advertising is approached as they create new strategies. The new growth in content marketing is usually perceived as a primary beneficiary of big data, although the concept of content marketing could be older than the internet itself. \\
Another essential point is that big data enables digital marketers to target users effectively with more personalized advertisements which they might prefer to see. Facebook and Google are among the biggest players in this domain of digital advertising. They have discovered excellent ways of creating and delivering more appealing advertisements in ways that do not intrude on the rights and preferences of the consumers  \cite{MangoldandFaulds2009}. Most of their advertisements feature services and goods that consumers would like most to enhance their lives and almost all of these advertisements are reliant on huge amounts of personal data that users usually provide from what they are up to, what they share and like things online.  \\
Experts contend that it is possible to accurately make predictions on an array of individual attributes that are more sensitive merely through an analysis of an individual’s Facebook or Twitter likes \cite{Nate2014}. For example, the likes on these social media websites are critical in predicting one’s religion, sexual orientation, emotional stability, life satisfaction, age, relationship status, and many other attributes. Companies like Facebook successfully linked political activity with user commitment when they created a sticker enabling users to declare on their profiles that they had voted. The initiative was conducted during the 2010 id term polls and was very effective. Individuals who saw the feature had high chances of voting and actively engaged in a conversation about the same after seeing their friends and peers participate in the activity. Later on, during the 2016 polls, Facebook escalated their role into the voting process by providing users with not only constant reminders but also with directions about their polling stations.

\section{USER RATING AND POP UP ADS}
Depending on the user preferences and the content that they often access on social media, pop-up advertisements can be created to target users every time they are online. For example, an ad can be created on the Facebook Messenger app to open inside that particular app every time the user hits the CTA button. When clicked, such adverts would redirect the user to a page where they would be required to answer some question, claim a reward or send some feedback regarding a product or service. Before creating such ads, it is imperative to establish a custom audience of the individuals who would be targeted with that particular pop-up ads. For instance, individuals who have previously liked the company’s products on their Facebook page or other social media sites can be included on the list of target audience to receive the ad \cite{Aycock2010}. Another strategy that can be employed is to rate users by tracking their cookies. In most cases, user activities are usually tracked across the internet using cookies whenever a user logs into one of the social media sites and is concurrently browsing other sites. Whenever this happens the other sites that the users is visiting can be easily tracked and the data used accordingly.

\section{CONCLUSION}
In summary, big data and social media have revolutionized advertising. Because of this, many organizations are currently harnessing the power of social media and big data analytics and creating marketing campaigns that target specific consumer needs. In the same way, different types of data generated from user interaction in various ways on social media platforms, such as comments on a status, likes, retweets, or shares, generate big data and enable the use of big data analytics. As a result, the data generated gives more insight about a user’s behavior and can be used to create marketing campaigns that specifically target users. Hence, this approach has been proven to be effective and most efficient enabling most organizations to experience an increased customer growth and a boost in overall sales.

\begin{acks}

The authors would like to thank Dr. Gregor von Laszewski for his support to write this paper as well as TAs' helpful suggestions on this paper. 

\end{acks}

\bibliographystyle{ACM-Reference-Format}
\bibliography{report} 

\end{document}