\documentclass[sigconf]{acmart}

\input{format/i523}

\begin{document}
\title{A Music Recommendation System}

\author{Shiqi Shen}
\affiliation{%
  \institution{Indiana University Bloomington}
  \streetaddress{1575 S Ira St}
  \city{Bloomington} 
  \state{Indiana} 
  \postcode{47401}
}
\email{shiqshen@indiana.edu}

\author{Qiaoyi Liu}
\affiliation{%
  \institution{Indiana University of Bloomington}
  \streetaddress{3209 E 10th St}
  \city{Bloomington} 
  \state{Indiana} 
  \postcode{47408}
}
\email{ql30@umail.iu.edu}

\begin{abstract}

Recently, people tend to use auto play-next while using a cloud-based music app. In this case, content recommendation is at the heart of most subscription-based media stream platforms. A good recommendation system based on a huge number of historical records and metadata can vastly enhance user experience and increase user engagement. In our project, we would build machine learning models to predict what songs are going to be listened next week. The training data would be a copy of user listening history during a month period and predict what songs a set of predetermined users would listen to during the next week. 

\end{abstract}

\keywords{i423, hid109, Big Data; Amazon; Customers; Pricing; Dynamic, Internet, application}

\maketitle

\section{INTRODUCTION}

The ability to generate and exchange information has increased tremendously over the recent past. This growth is driven by the easy availability and affordability of the computing as well as the ubiquity of the internet \cite{BergerandDoban2014}. In the current businesses world, almost everything is conducted electronically. There is a lot of information exchange and engagement over the internet, as well as selling and buying of products. Amazon is one of the leading giants in the application of big data. The firm is one of the pioneers of e-commerce, and one of its most outstanding innovations in this domain is the personalized recommendation system. The foundation of the system is big data, which is usually collected from the customers.  The firm has received various coveted awards due to its excellent innovations and application of big data \cite{BergerandDoban2014}. The firm has leveraged big data in the recent past to enhance its performance as well as service delivery to the customers. Together with other major firms in the internet services industry, Amazon acknowledged the significance of big data in the initial years of 2000, and then immediately focused on adequately using the big database of clients shopping on its online platforms. \\
Big data operates on the concept of the power of suggestion, as fronted by psychologists. They claim that by putting something that an individual may like in front of them, then they may have strong desire to purchase it. Amazon employed this philosophy by leveraging their customer data and transforming its system into a high powered one that is focused on the customer. The firm’s systems have been getting better by the day and expected to be even more superior in the near future.

\section{PRODUCT RECOMMENDER SYSTEM}

In the recent past, Amazon has moved from operating as a pure e-commerce firm to a major player in the internet services industry, with focus on offering a wide variety of services to both individuals as well as companies. The firm started to shift its focus on big data and started the journey to transition from a typical online retailer into one a major force in the realm of big data. Around 2000, the company, along with other internet firms such as Google, Yahoo, and Twitter realized that they had voluminous data about their customers, which could be put used to improve their performance. Although the other firms did not initially concentrate majorly on big data, Amazon swiftly moved to take advantage of the invaluable database of individuals who used its e-commerce platforms around the world to shop. The team charged with the responsibility of recommending the products to the customers came up with innovative strategies that the firm could make use of the data collected by the firm about their customers. The end result of the move was a huge success in big data, which revolutionized how the company did business. \\
As a major player in the e-commerce domain, the success of Amazon was always pegged on availing the right products to the customers. The efficacy of providing the right products for the customers in turn largely depended on a proper understanding of the needs of the consumers. A proper market research was necessary in order to understand the customer’s needs and tastes. Since it was founded, Amazon has created a name for itself because of its superior product recommender system, which suggests products to consumers on the basis of their last purchase. The major driving force behind the recommender system is the data gathered from the customers.  \\
The product recommender system is essential for the personalization of each customer’s experience when they are shopping in the firm’s online store \cite{Chen2012}. The firm employs collaborative filtering and clustering algorithms to classify clients on the basis of preferences.  Customers are grouped on the basis of same search as well as collaborative filtering between items. Content-based search employs the shopping history of customers and item ratings to establish a search query capable of finding other items that match the tastes of consumers. For instance, if a customer purchases a book, the product recommender systems will suggest books from the same author, publisher, or subject area. The product recommendations are not only used by the company in the online stores, but it also doubles up as a marketing tool useful in conducting email campaigns. There is a recommendation link that enables shoppers to filter products by several criteria depending on the items that they have in their shopping carts. 

\section{BIG DATA FOR DYNAMIC PRICING}

Dynamic pricing entails the use of big data such as clickstreams, purchase history, cookies, etc. to offer customized discounts to customers or to alter the prices of items being sold dynamically. The technology enables the real-time price customization for an item to suit a specific customer. This explains why it is sometimes possible for two different sets of customers to buy the same item at different prices from the same online store \cite{Benjamin2014}. Despite the immense benefits of this technology, some customers may always feel discriminated against due to the price differences. Amazon has successfully used the power of big data to implement a price discrimination system. For example, there was an incident in which some Amazon customers were aggravated about price variations of a certain DVD. One of the customers noted that there was a difference of nearly two points five dollars in the price if the cookers were deleted from the computer. Price discrimination was also experienced in the sale of a product known as Diamond Rio MP3 Player.  \\
Big data also enables price optimization. This enables the firm to manage the prices of commodities and grow its profits by twenty-five percent annually. Several factors are used to set the prices of commodities. Some of them are: activity of the customer on the firm’s shopping portal, availability of the product, competitor’s prices, order history, item preferences, and the anticipated profit margin \cite{Benjamin2014}. The prices are normally refreshed every ten minutes as big data become updated. Due to this, Amazon provides customers with discounts on best-selling commodities and accrue large profit margins on the items that are less popular with customers.

\section{BIG DATA AND CUSTOMER SERVICE}

Big data is also extensively being used for customer service at Amazon. The acquisition of Zappos has often been viewed as a major element in the same. Since it was founded, Zappos has enjoyed a good reputation for the excellence in customer service and was usually viewed as a world leader in this domain. Much of the success can be attributed to their advanced relationship management systems which extensively employed their own customer data. After the acquisition of the form in 2009, the procedures were integrated together with those of Amazon.  Today’s business environment is changing at a rapid rate, and consumers are also using their voices faster. Within a few moments after undergoing a bad experience, customers can swiftly move into social media and spread the news about their negative experience \cite{Bryant2008}. The only strategy for an organization to survive under such conditions is to employ the power of analytic to streamline and shorten the response time, as well as fix the customer support issues.  The customers of the present day are not only looking for a product that works, but also one that is personalized and able to recognize their interests and save them time.

\section{ONE CLICK ORDERING}
Amazon used big data to create one-click ordering. This feature is activated automatically when the customer places his first order, enters a shipping address as well as a method of payment. When using the one-click feature, the customer is given thirty minutes to change his mind about the particular purchase. This system was created on the premise that a simplified path to purchase would increase conversion rates. Since the introduction of the technology, the firm’s revenues have increased year after year. The significance of this application pushed the company to patent it to prevent other companies from using it without authorization. Reorganizing the purchase process is currently one of the most significant differentiates in the current marketplace.  The service enables users to make payments without having to exchange cards or money physically. Amazon has also greatly benefited from impulse buying, which is accelerated by one-click buying. Research has shown that the largest percentage of people normally purchase things they don’t require or did not plan to purchase in the first place \cite{Roy2002}.

\section{USING BIG DATA TO SUPPORT OTHER COMPANIES}
Amazon also uses its big data platform to support and help other companies improve their operations. Organizations can employ AWS toolkit provided by Amazon to create scalable big data applications that have the capacity to improve business performance \cite{Chen2012}. Besides, they would be able to secure these applications easily without the need to spend on expensive infrastructure and hardware. The big data applications including data warehousing, clickstream analytic, fraud detection, internet of things, and several others are delivered via cloud computing. Hence, there is no need for an organization to incur additional costs in setting up a data center. The Amazon web services can enable companies to analyze spending habits, customer demographics, and other related information to enable them effectively cross-sell some of the firm’s products in patterns similar to Amazon. That is to say that the retailers will also be able to stalk their customers, recommend products to them, and improve their customer experience.

\section{BIG DATA TECHNOLOGIES}
Amazon EMR: This technology offers a managed Hadoop framework that simplifies and hastens the processing of huge amounts of data across scalable Amazon EC2 instances. Amazon EMR also supports other common distributed frameworks including HBase, Apache Spark, Flink, and Presto \cite{Amazon2017}.  Besides, it reliably and safely handles a wide range of big data use cases, such as web indexing, log analysis, financial analysis, machine learning, and bioinformatics. \\
Amazon Athena: It denotes an interactive query service that simplifies data analysis in Amazon S3 via standard SQL. Since it is service less, one only pays for the queries they run and there is no infrastructure to be managed \cite{Amazon2017}. The technology is quite straightforward and delivers results within the shortest time possible. Moreover, it does not require complex ETL jobs to prepare data for analysis. \\
Amazon Kinesis Firehouse: This is one of the simplest methods to import streaming data into Amazon Web Services. The technology can be used to gather, transform, and import streaming data into Amazon S3, Amazon Kinesis analytic, and Amazon Redshift, to permit instant analytic with the current BI tools and dashboards currently being used.  It is a comprehensively managed service that can expand automatically with the increase in data throughput. 

\section{CONCLUSION}
Big data has grown tremendously in the recent past. The growth has been accelerated majorly by the increased accessibility of computing devices as well as the ubiquity of the internet. Being one of the pioneers of e-commerce, Amazon has extensively employed big data to improve its performance. Big data has been used to create recommender systems, implement dynamic pricing, streamline and improve the customer experience, and support other companies. The system recommends products to customers based on their purchase history and enables them to filter the products list based on certain criteria. The company continues to enhance its big data applications with a view to creating a loyal customer base. 

\begin{acks}

The authors would like to thank Dr. Gregor von Laszewski for his support to write this paper as well as TAs' helpful suggestions on this paper. 

\end{acks}

\bibliographystyle{ACM-Reference-Format}
\bibliography{report} 

\end{document}