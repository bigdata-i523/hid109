\documentclass[sigconf]{acmart}

\usepackage{graphicx}
\usepackage{hyperref}
\usepackage{todonotes}

\usepackage{endfloat}
\renewcommand{\efloatseparator}{\mbox{}} % no new page between figures

\usepackage{booktabs} % For formal tables

\settopmatter{printacmref=false} % Removes citation information below abstract
\renewcommand\footnotetextcopyrightpermission[1]{} % removes footnote with conference information in first column
\pagestyle{plain} % removes running headers

\newcommand{\TODO}[1]{\todo[inline]{#1}}

\begin{document}
\title{Diversification of Big Data}

\author{Shiqi Shen}
\affiliation{%
  \institution{Indiana University Bloomington}
  \streetaddress{1575 S Ira St}
  \city{Bloomington} 
  \state{Indiana} 
  \postcode{47401}
}
\email{shiqshen@indiana.edu}

\author{Qiaoyi Liu}
\affiliation{%
  \institution{Indiana University of Bloomington}
  \streetaddress{3209 E 10th St}
  \city{Bloomington} 
  \state{Indiana} 
  \postcode{47408}
}
\email{ql30@umail.iu.edu}

\begin{abstract}

There are some ideas around the conception of big data and how it is used in the market outside of a programmer’s computer. What I mean to try, and state here is that there is an idea of how big data is used and functions in the world. As a programmer, we cannot just focus on what we contribute to the programs, meaning we cannot only look at our computer data and creation of such programs as a means for our own scientific endeavors and egos, we must understand how such creation are reflected into the world and how such programs create a very interesting market exposure. Therefore, we write this paper to analyze the enterprise of big data. We will use this paper to seek out the multiplicity of avenues in which big data is used by our technological world (mainly those that seek to use big data to create diversified consumer experiences). 

\end{abstract}

\keywords{i423, hid109, Big Data; Social Media; Online, Shopping, Customers; Pricing; Dynamic, Internet, application}

\maketitle

\section{INTRODUCTION}

In order to begin a type of observation around this topic, we need to first establish what types of enterprise we will look into in order to make our observations. We will look at Online Shopping, streaming services, and Social media. These forms of enterprise concentrate their use of big data to create a diversified user experience, one which looks to creating more possibilities and other forms of consumer products for consumers. We would also like to use such an observation to understand the use of big data in these types of enterprises and how such uses render experiences and technology of big data as a good form of consumer study.

\section{OBSERVATION}

Before we begin our in-depth observation, it is important to showcase the Big Data technologies, some components of big data that make it what it is. There is of course an array of technologies that facilitate and create big data. In the creations of big data, we can see many enterprises like the generations of web pages (in which individuals, corporations, government, and the like, produce these pages with data). We can also see digital imagers that facilitate data collection as well. These types of data can come from telescopes, MRI Machines, and Video Cameras. Another source of data production can come from biological and chemical sensors, things like microarrays and environmental monitors \cite{3}. \\

From production of data there must be a medium that collects this data, that is of course usually computers. These data can be collected in things like the internet and localized sensor networks. These sources of course can both collect and analyze the data. From collection, there are also storage capacity for these data. In that, we find that there are many storage type disk (magnetic disk) that can hold tons of data. There are also cloud services with which data can be stored.  \\

Big data is crucial to the developments of many businesses that drive on the interactions and recommenders of their clientele. When we are looking to understand big data and how it is used by big contenders, we must first look to the companies to see what uses they have for big data. Let us begin with looking at Amazon, an empire built on e-commerce, the selling and recommending of products to consumers at efficient and effective rates. When we look at how this company succeeds, we can see that the main component to its success comes from its unique and bold use of big data. 

\section{ALGORITHMS, PREDICTIONS, AND BIG DATA CRAZE WITH ONLINE SHOPPING}

When a consumer is shopping on Amazon, they can click on the image of the product they would like to view. During this process, they will probably go back and look at other products to compare the product to other products that are similar. Therefore, what can a company do to ensure that they are helping their consumers with this process? By pooling this data into other similar consumer searches and sells. For Amazon to do this they have a very complex system of enterprise that is meant to survey this, their use of big data. But it also comes down to very simple standards that are used by consumers.  \\

When a consumer is shopping on Amazon they can administer filters to help them narrow their searches of products that they like. But during this process of administering these filters, the consumer is also being surveyed of searches made in the past to better associate what the consumers wants and needs within the products that they are shopping for.\\

This type of use of big data allows for any company (mainly Amazon like companies) to focus their resources on the calculation of products that an individual consumer would want based on searches that they have made. But the best part about this is that the information is pooled on multiple similar searches and consumers to best devise the necessary searches to show a consumer what to buy \cite{1}.\\

But beyond this simple observation of the big data usage, we need to understand what forms of implementation must be taken into consideration when creating these types of programs. Now, what really needs to be focused on here is the interpretation of the data that is collected by Amazon to provide consumers with their recommendations.\\

Within big data itself, the name gives away what it is, it’s a cluster of date, a large, almost seemingly insurmountable amount. There must be some sort of way to interpret the data results as they come in. For this process, it needs to be understood that this data interpretation cannot happen within a void, rather what is done for these data to be interpreted and re-designated as recommendations back to a user would be through a process of examining detailed assumptions and retracting the analysis \cite{2}.\\

This process of observing and interpreting big data itself can have issues, such as those of bug interference based on programs that are being used in order to interpret these large data pools, and data can become erroneous. However, a way in which these types of issues can be resolved comes in the form of predetermined data assumptions. Data assumptions that are made to help companies who use big data to narrow in on data pools to create a seamless connection of data gathered to products showcased based on data that is collected, interpreted, and re-designated to users. This is done to help devise the necessary sales goals or recommendation services that come from these companies use of big data.\\

Also, when we are looking at the process by which data is collected, there will be data that is of no importance to the necessary processes by which the data is collected for. The difficult task becomes that of filtering out the useless information without removing the information that is of importance. To do this, there have been advancements made within the scientific community to reduce the plausibility of such turn out to happen. The process seeks to monitor faultiness that can be caused by sensors to lower the chances of data that is useful to not be discarded alongside data of no importance. This is also where algorithms that are made to establish key components of data come into play as well. \\

These forms of interpretation of course come from algorithms that are used to detect the data in forms of patterns. A perfect example of this would be comparing both Amazon and Netflix’s use of algorithms that recommend to users what to buy or watch next. Machine learning within the recommendation system is what enterprises this use of algorithms. The machine learning itself will compare the histories of purchase and views to establish a statistical model of a collective pool of millions of other users to generate the necessary continued recommendations to users. The use of algorithms and machine learning also helps to establish a base line of what it is that users like and are more attuned. The use of algorithms is to ensure that the data that is being collected is correctly sorted and re-designated to users in the forms of recommendations. This is because large quantities of data require these types of “assumptions” to calculate and extract knowledge from the data that is collected \cite{3}.\\

As a programmer, machine learning and the creation of these algorithms truly fascinate when they are being used in order to create a recommendation system. Since the collection of data is pooled, the use of algorithms to reach a particular end, based on data analyzation, makes for a very interesting usage. As the use of data algorithms not only seek to facilitate the necessary recommendations, but also filter through millions of data informatics to agitate the necessary products to ensure the transparency of recommendations created. This can also be seen in what was stated above regarding consumers being able to filter products with precise search options. These filters can act as filters for filters. As in the filters chosen by consumers help to facilitate precise sifters and allows for algorithms to pinpoint precise outlines of data that help to recommend even finer tuned recommendations for consumers. \\

Big data, from this perspective, can be seen as a general tool that can be created to make precise measurements that are used in order to facilitate the necessary recommendations for consumers. This is of course a process that is keyed out by the use of interpretations and algorithms that are necessary to the specifications made on big data that is collected. Therefore, big data is used precisely well as a tool to pool and narrow pattern like data to ensure that completed data of particular patterns can always correctly correlate to consumers as well as viewers who use services such as Amazon and Netflix (This paper will not observe Netflix in full as it did with Amazon since the use of big data analyzes and processes to creating recommendations for consumers are very similar).\\

Therefore, when we are looking at companies like Amazon and Netflix, we are looking at companies that use big data to create predictive analytics. Predictive analytics has many uses however and cannot just be subjected to Amazon and Netflix’s use on commercial needs. The use is mainly attributed to uncovering patters and highlight relationships with the data that is being observed. Because of this very nature, big data that goes under predictive analytics are being done son to search out these two main components. There is also the process of trying to find past data patterns outcome variables and trying to deduce them for the use of the future (observing patters from a specific time-period to see if such trend continues again at another given future based on certain functions that existed in the data of the past and then comparing that to the future) \cite{4}. \\

Further delving into predictive analytics, which is the main use of big data for our commercial subjects, we see that there are also for of linear regressions. The use of this is to find interdependencies within outcome variables and explanatory variables in order to use them in the process of making predictions or to right out make the prediction itself. This looks to focus the data that is collected into predictive measures that are precise to the sets of data that are collected and distributed through this type of analysis. Above, when looking at machine learning, this is categorized as a neural network. Neural networks are a collective entity, something like that of the human brain, “An artificial neural network can be defined as a computing system made up of number of simple highly interconnected processing elements which processes information by their dynamic state response to external inputs” . Within the neural networks, machine learning works within the sphere of the networks to generate and learn from data collected to predict, showcase patterns, and classifying input data \cite{5}.\\

To finish up the observation here on big data uses for predictive analytics and the basis of enterprise that is that of Amazon and Netflix’s ability to use big data to create such systems, the paper needs to understand a little further into the principles of how this portrays use in consumer settings. What is meant to be clarified here is how big data effects the process by which consumers use services and are data mined. To affectively understand this component, there needs to be an understanding that big data and it cultivation are just as it is named, a collection of data on a mass scale meant to just be data. It is however, the enterprise of the scientific community as well as commercial bodies that induce a specific plethora of uses to assimilate the necessary components to use big data effectively. Such things make it to where consumers have an easier time with their collective use of these commercial branches (Amazon and Netflix, as well as unnamed companies). Consumers use of these sites creates data, a process by which these companies then use the process of data mining to achieve the best standards of prediction and analytics that help them to enforce their use of big data as their tool in apprehending consumers by prediction. Because of this process, consumers are the ones who are helping these companies further develop their own uses of big data by allowing these corporate bodies to data mine them, collect data, and analyze the data provided. But this process is crucial to the experience of the consumer, as this data helps with creating the necessary components of these corporate bodies, allowing them to further create advanced algorithms that help these corporate bodies devise the necessary recommendations and make the predictions to the habits of these consumers. That makes it easier for consumers to use these products offered by these corporate bodies.\\

In all, big data, with modifications from things such as data mining, data collection, machine learning, algorithms, and prediction analyst are all components which excel the use of big data (because of the necessary enterprise that it takes to stay updated with this matter) and insure the that consumers and users of big data can reach out to their own platforms easily. \\

\section{PRODUCT RECOMMENDER SYSTEM}

In the recent past, Amazon has moved from operating as a pure e-commerce firm to a major player in the internet services industry, with focus on offering a wide variety of services to both individuals as well as companies. The firm started to shift its focus on big data and started the journey to transition from a typical online retailer into one a major force in the realm of big data. Around 2000, the company, along with other internet firms such as Google, Yahoo, and Twitter realized that they had voluminous data about their customers, which could be put used to improve their performance. Although the other firms did not initially concentrate majorly on big data, Amazon swiftly moved to take advantage of the invaluable database of individuals who used its e-commerce platforms around the world to shop. The team charged with the responsibility of recommending the products to the customers came up with innovative strategies that the firm could make use of the data collected by the firm about their customers. The end result of the move was a huge success in big data, which revolutionized how the company did business. \\

As a major player in the e-commerce domain, the success of Amazon was always pegged on availing the right products to the customers. The efficacy of providing the right products for the customers in turn largely depended on a proper understanding of the needs of the consumers. A proper market research was necessary in order to understand the customer’s needs and tastes. Since it was founded, Amazon has created a name for itself because of its superior product recommender system, which suggests products to consumers on the basis of their last purchase. The major driving force behind the recommender system is the data gathered from the customers.  \\

The product recommender system is essential for the personalization of each customer’s experience when they are shopping in the firm’s online store \cite{1}. The firm employs collaborative filtering and clustering algorithms to classify clients on the basis of preferences.  Customers are grouped on the basis of same search as well as collaborative filtering between items. Content-based search employs the shopping history of customers and item ratings to establish a search query capable of finding other items that match the tastes of consumers. For instance, if a customer purchases a book, the product recommender systems will suggest books from the same author, publisher, or subject area. The product recommendations are not only used by the company in the online stores, but it also doubles up as a marketing tool useful in conducting email campaigns. There is a recommendation link that enables shoppers to filter products by several criteria depending on the items that they have in their shopping carts. 

\section{BIG DATA FOR DYNAMIC PRICING}

Dynamic pricing entails the use of big data such as clickstreams, purchase history, cookies, etc. to offer customized discounts to customers or to alter the prices of items being sold dynamically. The technology enables the real-time price customization for an item to suit a specific customer. This explains why it is sometimes possible for two different sets of customers to buy the same item at different prices from the same online store \cite{Benjamin2014}. Despite the immense benefits of this technology, some customers may always feel discriminated against due to the price differences. Amazon has successfully used the power of big data to implement a price discrimination system. For example, there was an incident in which some Amazon customers were aggravated about price variations of a certain DVD. One of the customers noted that there was a difference of nearly two points five dollars in the price if the cookers were deleted from the computer. Price discrimination was also experienced in the sale of a product known as Diamond Rio MP3 Player.  \\

Big data also enables price optimization. This enables the firm to manage the prices of commodities and grow its profits by twenty-five percent annually. Several factors are used to set the prices of commodities. Some of them are: activity of the customer on the firm’s shopping portal, availability of the product, competitor’s prices, order history, item preferences, and the anticipated profit margin \cite{Benjamin2014}. The prices are normally refreshed every ten minutes as big data become updated. Due to this, Amazon provides customers with discounts on best-selling commodities and accrue large profit margins on the items that are less popular with customers.

\section{BIG DATA AND CUSTOMER SERVICE}

Big data is also extensively being used for customer service at Amazon. The acquisition of Zappos has often been viewed as a major element in the same. Since it was founded, Zappos has enjoyed a good reputation for the excellence in customer service and was usually viewed as a world leader in this domain. Much of the success can be attributed to their advanced relationship management systems which extensively employed their own customer data. After the acquisition of the form in 2009, the procedures were integrated together with those of Amazon.  Today’s business environment is changing at a rapid rate, and consumers are also using their voices faster. Within a few moments after undergoing a bad experience, customers can swiftly move into social media and spread the news about their negative experience \cite{Bryant2008}. The only strategy for an organization to survive under such conditions is to employ the power of analytic to streamline and shorten the response time, as well as fix the customer support issues.  The customers of the present day are not only looking for a product that works, but also one that is personalized and able to recognize their interests and save them time.

\section{ONE CLICK ORDERING}
Amazon used big data to create one-click ordering. This feature is activated automatically when the customer places his first order, enters a shipping address as well as a method of payment. When using the one-click feature, the customer is given thirty minutes to change his mind about the particular purchase. This system was created on the premise that a simplified path to purchase would increase conversion rates. Since the introduction of the technology, the firm’s revenues have increased year after year. The significance of this application pushed the company to patent it to prevent other companies from using it without authorization. Reorganizing the purchase process is currently one of the most significant differentiates in the current marketplace.  The service enables users to make payments without having to exchange cards or money physically. Amazon has also greatly benefited from impulse buying, which is accelerated by one-click buying. Research has shown that the largest percentage of people normally purchase things they don’t require or did not plan to purchase in the first place \cite{Roy2002}.

\section{USING BIG DATA TO SUPPORT OTHER COMPANIES}
Amazon also uses its big data platform to support and help other companies improve their operations. Organizations can employ AWS toolkit provided by Amazon to create scalable big data applications that have the capacity to improve business performance \cite{1}. Besides, they would be able to secure these applications easily without the need to spend on expensive infrastructure and hardware. The big data applications including data warehousing, clickstream analytic, fraud detection, internet of things, and several others are delivered via cloud computing. Hence, there is no need for an organization to incur additional costs in setting up a data center. The Amazon web services can enable companies to analyze spending habits, customer demographics, and other related information to enable them effectively cross-sell some of the firm’s products in patterns similar to Amazon. That is to say that the retailers will also be able to stalk their customers, recommend products to them, and improve their customer experience.

\section{BIG DATA TECHNOLOGIES}
Amazon EMR: This technology offers a managed Hadoop framework that simplifies and hastens the processing of huge amounts of data across scalable Amazon EC2 instances. Amazon EMR also supports other common distributed frameworks including HBase, Apache Spark, Flink, and Presto \cite{Amazon2017}.  Besides, it reliably and safely handles a wide range of big data use cases, such as web indexing, log analysis, financial analysis, machine learning, and bioinformatics. \\

Amazon Athena: It denotes an interactive query service that simplifies data analysis in Amazon S3 via standard SQL. Since it is service less, one only pays for the queries they run and there is no infrastructure to be managed \cite{Amazon2017}. The technology is quite straightforward and delivers results within the shortest time possible. Moreover, it does not require complex ETL jobs to prepare data for analysis. \\

Amazon Kinesis Firehouse: This is one of the simplest methods to import streaming data into Amazon Web Services. The technology can be used to gather, transform, and import streaming data into Amazon S3, Amazon Kinesis analytic, and Amazon Redshift, to permit instant analytic with the current BI tools and dashboards currently being used.  It is a comprehensively managed service that can expand automatically with the increase in data throughput. 

\section{UNSTRUCTURED DATA AND AI COMPONENTS OF SOCIAL MEDIA}

Next, we will look to observe big data and its uses within social media. First and foremost, it is important to understand that social media is an outlet that is massive. The many posts, tweets, likes, shares, and other social media actions all develop unstructured forms of data that are then considered by corporations to understand the market of users. It is within the creation of this unstructured data that creates such an importance to talking about social media and its perfect relationship to big data. Because of the importance of social media to businesses (due to trend and the fast-paced living environment we live in, if a business is behind on trend it becomes behind on sales and other forms of innovations), there is a large component of dependence towards retrieving big data from social media to calculate and predict trend. But beyond just trend, business are also trying to get their hands on the enormous amount of big data that exists within the social media sphere. There is recognition of the value of unstructured data that is sourced within social media. The value comes from consumers using social media to broadcast what they are thinking, want, and are doing. These types of information, one might think it private, but the internet is a very transparent source of data. In so, businesses value the perspective of consumers and create ways in which they can follow through with interacting on a very contingent basis, they data mine and from that, they advertise based on the collected big data from social media \cite{6}.\\

This brings to focus the use of advertisement and the necessity for big data to be used. Within digital advertising, the one who collects and analyzes big data effectively and efficiently with accurate uses is king. To be successful in this method of advertising, businesses need to be prodigious at their collection of data, integration of that data and analyzation of that data. The reason being is if these three things are managed well, the use of the data is much more successful. What is challenging about this type of work however is that a majority of the data that is collected from social media is unstructured, it is in a word, messy. These forms of unstructured data are in our own use of social media, usually within posts, videos, tweets, photo post to Instagram, mass use of Snapchat, etc. Because these forms of data are so much more unstructured and more difficult to analyze using traditional analyzation methods, businesses needed to enterprise methods for them to collect these data forms and have the necessary big analytics platforms to analyze the data. Keep in mind the data has the most information about us, therefore the use of these unstructured data is key to devising targeted and precise marketing executions \cite{7}. \\

There is also the collection of real-time data. Because of the advances within the technology, the process by which real-time data can be analyzed has sped up exponentially compared to the past where this process would be impossible and yield no results. With the ability to now look real-time data and have the capacity to analyze it, businesses (those that are marketers) have the possibility of taking acting instantly to provide consumers on social media with their own personalized ad. The use of personalized ads of course comes from the massive amounts of data that consumers put out on social media, allowing marketers to collect these data (things we like, talk about, and do daily). And because they have these data, they can target specific ads to consumers without missing a beat. But what happens when we being to incorporate other technologies that allow us to always be able to access our social medias? Mobile devices provide the quintessential provisions needed for data to have a constant flow to advertisers. Big data then is a facet within the life of social media. It must be done effectively in order to continue its uses of data collected and then expounded on to get customers to buy products or even interact with specific businesses. With the developed of the mobile device, location also becomes part of big data, as it allows for your location to be collected, you leave not only a digital foot print, but also a physical one which can be collected as data and used \cite{7}.\\

Because of such enterprise, big data can gather almost every facet of information that is readily available to the internet. Such a mass look on data also comes back to the revision on algorithms that are meant to specify what is being observed and analyzed in the data. Of course, for these algorithms to work, there has to be a steady stream of data in order for the algorithms to process and do its job. Because of the accessibility of data through the means of mass social media and how quickly consumers use and are exposed to social media (due to mobile phone), businesses are more creative in their approach to their algorithms. These algorithms can now pop up wherever consumers are on the social sphere. In doing this, big data itself changed the advertising platform. Advertiser now must create enticing messages from big data to continue their reach to consumers. The change is incredibly eye opening. As the use of data itself can create a massive alteration in how a marketer begins to try and craft their ads to highlight what it is that individuals want. Advertising becomes more individualized and work closely around the sphere of social media to allocate their ads effectively \cite{8}.\\

Within the use of big data, there is also the use of AI to help with the process of analyzing big data. AI creates a much more effective measure when it comes to analyzing hundreds and thousands of data in detail. Because AI can do that, it allows for businesses to have a better idea around what perspective they themselves must take when advertising based on detail production from AI technology. AI technology becomes a very important component to being able to go analyze big data in order to provide the necessary specific information that would help with creating the necessary ads \cite{6}. \\

We are then looking to see how this affects the user, or better how this type of big data in social media affects consumer experiences. The use of big data comes back to the work of the consumer and the business. The consumer creates big data from their usage of social media, social media in turn collects and responds to the data that is being created by the consumer (user). Through the process of facilitating the data, analyzing it, interpreting it, pooling it, and filtering it through algorithms and AI technology, consumers using social media get access to tailored advertisements. The consumer experiences a personalized social media experience and personalized advertisement trail based on the collected data that is reinforced by big data that is pooled together in order to do this. Does big data have any other uses in social media then? Yes, it does. \\

Social media can also use big data in order to create studies and infiltrate certain components of a consumer’s private life. Facebook for example, a top social media company prides itself in its technological advances that use big data. Facebook uses is considered a top user of digital advertisement, so it begs to question what else does Facebook do with the mass big data that it collects? Facebook has also used its big data resources to try and act on social media experiments. During a time when there we the “I Voted” sticker sprawled on Facebook for it users to share and gimmick that they had voted, Facebook was using this in order to incite and boost voter turnouts. The method was to first isolate and use the stickers with particular groups of people, small groups at first in order to test the role. After having enough data collected on the presence of the stickers and what they meant for users, Facebook began to mass data span by incorporating the stickers in a much more massive turnout. With midterms of 2010, there is studied on behalf of Facebook scientist, that say 340,000 more people voted in the 2010 midterms \cite{9}. \\

From these the observation of these big data uses, we can showcase the how and what makes big data so important to the creation of diversified consumer experiences but also showcase the necessary components to how big data is used by the new technologies we have. It is however imperative that we understand these different phenomena that come from the use of big data.

\section{SOCIAL MEDIA IS SIGNIFICANT FOR COMPANIES AND INDIVIVUALS}

Although big data is said to come from several different sources, the largest proportion of it is said to originate from unstructured sources. As it can be imagined, social media makes up the largest source of unstructured content for big data. All the activities that users perform on social media such as views, retweets, comments, favorites, likes, etc. can be gathered and explored by interested individuals. \\

In the current digital world, social media plays a vital role in many companies. Having a presence on various social media platforms such as Instagram, Facebook, and Twitter is imperative since it enables individuals to interact with an organization on an ostensibly personal level and at the same time helps businesses across several domains get in touch with their customers. Currently, Facebook alone has over two billion users on their platform; this is roughly twenty-six percent of the world population \cite{6}. It is therefore important to consider the fact that big data, from the social media platforms, can reach any people in different forms. Besides that, social media interactions have continued to play a big role and will continue to play a big role in business decisions. For example, some insurance companies have declined to offer life insurance policies to individuals solely based on their social media posts. If you frequently post, on any of these platforms, about how you are drinking or going to drink, insurance companies would be reluctant to offer you a life insurance policy as this is a risk to them. \\

It will not be long before organizations discover new and better strategies for making sense of big data. But, at the moment, the concept of big data is still new and rapidly evolving. Nevertheless, some businesses have found ways of interacting and using this data, which is just but the beginning, but still a good way to begin. To elaborate, a marketing company whose interest is promoting a new product could employ machine learning algorithms that enable it to gather data from individuals who meet certain attributes \cite{6}. Consequently, by employing artificial intelligence technology, they will also be capable of drawing insights from millions of users and create campaigns. This will increase their levels of precision and focus, a technique usually referred to as targeted marketing, and present an excellent opportunity for finding the perfect audience and satisfy its preferences. 

\section{BIG DATA IN SOCIAL MEDIA ADVERTISING}

Fundamentally, advertising revolves around communication since it is all about sensitizing consumers on products and services that an organization is selling. However, different consumers will always want to hear varied messages, which is a vital fact to consider when new clients are being recruited into the internet bandwagon due to the growing popularity of smart phones. Big data has the capacity to customize these messages, project what consumers would like to hear, and establish new perceptions on what customers like or prefer \cite{HenselandDeis2010}. The above steps are all revolutionary and are expected to have a significant impact on how marketers in various organizations advertise.   \\

Furthermore, there are some occurrences which several people do not view as advertising but are still interactions between big data and marketing like product recommendation. An obvious example is Netflix \cite{8}. Although the company does not have a concrete advertisement plan, it employees a lot of algorithms to recommend various movies and shows to its customers. The approach saves the organization a lot of money by reducing the rate of customer exit and ensures that the right shows are marketed to the right individuals. The company’s strategy is to target consumers with shows specifically tailored for them. Apart from them, other firms such as Amazon, YouTube, etc. also do the same by using product recommendation to target their customers  \cite{8}. In order to stay up to date, the algorithms need constant flow of data to help it work more efficiently. With the growth of the internet, users leave huge volumes of data not only on social media platforms but also on other places they visit online in the form of a digital footprint.  This provides advertisers with new avenues to tailor their messages to meet their customer demands.  \\

The digital footprints left by online advertisers provides new insights to marketers on what a consumer really needs, and this sometimes may be more accurate than what the customer actually says on social media. However, marketers are worried about how to safeguard the privacy and security of their consumers and therefore companies that are careless in handling data collected from consumers usually ignite a backlash which greatly impact their business. Even though targeted advertising has been in existence for quite a while \cite{8} the more the data that is collected by advertisers, the more personalized and effective marketing is expected to be. Organizations will strive not just to gather as much data as they can, but also to gather information which typically represents the individual consumer’s needs in order to enable them to market to their personalized tastes.

\section{ANALYZING LINKS}

Big data collected from social media can lead to the discovery of new information regarding each individual customer that can help in creating a customized appeal to that specific customer. However, with the new insights, marketers can enhance how advertising is approached as they create new strategies. The new growth in content marketing is usually perceived as a primary beneficiary of big data, although the concept of content marketing could be older than the internet itself. \\
Another essential point is that big data enables digital marketers to target users effectively with more personalized advertisements which they might prefer to see. Facebook and Google are among the biggest players in this domain of digital advertising. They have discovered excellent ways of creating and delivering more appealing advertisements in ways that do not intrude on the rights and preferences of the consumers  \cite{MangoldandFaulds2009}. Most of their advertisements feature services and goods that consumers would like most to enhance their lives and almost all of these advertisements are reliant on huge amounts of personal data that users usually provide from what they are up to, what they share and like things online.  \\

Experts contend that it is possible to accurately make predictions on an array of individual attributes that are more sensitive merely through an analysis of an individual’s Facebook or Twitter likes \cite{7}. For example, the likes on these social media websites are critical in predicting one’s religion, sexual orientation, emotional stability, life satisfaction, age, relationship status, and many other attributes. Companies like Facebook successfully linked political activity with user commitment when they created a sticker enabling most of their users to declare on their profiles that they had voted. The initiative was conducted during the 2010 midterm polls and was very effective as more people turned up to vote as compared to the 2006 midterm elections \cite{9}. Individuals who saw the feature had high chances of voting and actively engaged in a conversation about the same after seeing their friends and peers participate in the activity. Later on, during the 2016 polls, Facebook escalated their role into the voting process by providing users with not only constant reminders but also with directions about their polling stations \cite{Perez2016}. Apart from that, they also enabled users to easily get access to registration information, news, voting guides and other tools that would have made them more equipped to go through the election process.

\section{USER RATING AND POP UP ADS}
Depending on the user preferences and the content that they often access on social media, pop-up advertisements can be created to target users every time they are online. For example, an ad can be created on the Facebook Messenger app to open inside that particular app every time the user hits the CTA button. When clicked, such adverts would redirect the user to a page where they would be required to answer some question, claim a reward or send some feedback regarding a product or service. Before creating such ads, it is imperative to establish a custom audience of the individuals who would be targeted with that particular pop-up ads. For instance, individuals who have previously liked the company’s products on their Facebook page or other social media sites can be included on the list of target audience to receive the ad \cite{Aycock2010}. Another strategy that can be employed is to rate users by tracking their cookies. In most cases, user activities are usually tracked across the internet using cookies whenever a user logs into one of the social media sites and is concurrently browsing other sites. Whenever this happens the other sites that the users is visiting can be easily tracked and the data used accordingly.

\section{RELEVANCY OF BIG DATA ANALYTICS IN GROCERIES STORES}

\subsection{Increases the customer shopping experience }
As per a current SHSFoodThink white paper "Are We Chain Obsessed?" 64{\%} of customers said that the previous shopping experience is what makes them keep coming back—not the items themselves \cite{12}. By utilizing bits of knowledge received from the information transaction database, online networking, promotional activity, customers purchasing behavior, and client movement patterns, grocery stores can find a way to guarantee they are engaged with their customers that matter most.  
\\
	 For instance, they can investigate customers shopping movement to enhance the layout of their store, or recognize attrition risks for clients who have not as of late bought staple things, similar to milk. In like manner, chains can construct item varieties demonstrated with the customer needs and purchase patterns in certain regions \cite{10, 12, 14}. Regardless of whether it is through reconsidering store layout or furnishing store attended with mobile apps to better serve clients, analytics can enable grocers to change consumer’s expectations. 



\subsection{RESTRUCTURE THE SUPPLY CHAIN}

Grocery stores can likewise utilize analytic to investigate the production of their products, monitor production processes, and quality control, and improve straightforwardness with buyers about their sustenance production practices of foods \cite{11}. Suppliers remain to profit from the evaluation also, with access to secure, customized content of information identified with performance sales of the product, stock, margins, and marketing effectiveness. Giving supplier an opportune profitable business knowledge that supports joint ventures, drives performance, and decreases waste products

\subsection{BUILD SUPERIOR MARKETING PROGRAMS}

Loyalty programs furnish grocery merchants with an abundance of data to enable them to distinguish client segments and precisely characterize item preferences. By joining this information with different data sources—like healthful patterns, favored technique for accepting marketing promotion, customer movement patterns, and weather-related event—grocery merchants can concentrate on enhancing, and derive income from, the general shopping experience \cite{12}. For instance, grocery retailers can utilize analytics to customize the advancements they offer to clients given what they are well on the way to buy. They can likewise time advancements fittingly, and offer codes to customers who often as possible buy certain things. 

\subsection{IMPROVES HR STRATEGIES}

Supermarket stores utilize analytics to manage work-related decisions. Information freely accessible through online networking accounts and different means can be examined in conjunction with a grocer's internal information to direct decision identified with selection and recruitment, employee termination, and performance management and advancements \cite{13}. For example, an investigation of late action on LinkedIn can reveal insight into which representatives are destined to leave an organization. \\

Grocery merchants can likewise break down information to control the advancement approaches that will build workforce performance. For example, they could explore different avenues regarding organizing a social gathering for representatives at a subset of their stores, and analyze information on profitability, morale, and turnover in the preceding months \cite{10}. They may find that the gathering information prompted a more positive workplace where workers feel more noteworthy engagement at work, and soon after that, they could roll the strategy out to different stores. 


\subsection{USING BIG DATA FOR COMPETITIVE ADVANTAGE AND ATTRACTING CUSTOMERS}

Numerous grocery stores have been utilizing transaction and client information for a considerable length of time, despite the fact that many still have not completely used all that can be proficient with these types of information. For Small to Medium Sized grocery merchants, many have swung to subcontracted point solutions because of an absence of available analytics assets and potential framework investment required \cite{12, 13}. The issue with point solutions recently is that – they independently work out for a particular business section and the evaluation is cookie cutter. In this way, the 'information' is not coordinated and hard if not difficult to give an all-encompassing picture of client conduct overall touch focuses for instance.  Nor are the investigations offering a cross-functional observation that is pertinent to all business partners as far as driving differentiation in the commercial center in promoting, advertising, store operations and supply chain. \\

As far as utilizing 'new' data sources, for example, mobile, social and text, the industry is particularly occupied with a discovery' phase of investigation with an assortment of center sections, testing and figuring out how to extricate an incentive from these rich new sources of information. There are two common paths grocery merchants takes with little respect of the 'size' of the organization: to start with is Strategic Commitment, in which there is C-level (hierarchical) commitment making the venture in the assets to get the majority of the in-house data and evaluated it \cite{10}.\\

Presently like never before, information, analytics, and IP are seen as vital resources and competitive discriminators. The other is Business Discovery; in which grocery merchants outsource to an Analytics as a Service firm to use internal and external information. Performing analytics speeds the construction of business advantages creating new users case and helps catch 'quick wins' before making resource commitment to technological innovation and human capital in advance \cite{13}. In view of progress, and a wit, trusted stakeholder willing to share the techniques and explanatory models, can assist grocery merchants to proceed with an outsourced administrations supplier or relocate the data, analytics in addition to IP in-house. 


\section{RECOMMENDATIONS}
\subsection{Real-time insight on product demand}

Nowadays, retailers can get to information on item demand levels instantly on a chain of stores. Nevertheless, numerous merchants are still in the earliest stages in regards to evaluating and monetizing the huge amount accessible data \cite{14}. This prompts stocking deficits, for example, evaluating item demanded based exclusively on past historical information. It can likewise convey about wrong promoting endeavors: If a customer purchased ketchup on Saturday, an email coupon for it on Sunday is not well planned and make little sense to the shopper.  \\

This is the place data from store loyalty programs in addition to credit card sales can prove to be useful. Its data can be utilized to define needs of the customers in future. For example, grocery merchants can use data analytics to decide how regularly customers purchase sugar, flavors, or different items, and after that send every family unit coupons given their propensity to buy \cite{12}.



\subsection{Enhancing in-store stock management }

Perishable basic supplies, for example, dairy, meat, and fish call for precise stock administration, regularly on an hourly premise. Client analytics and prediction tools can enable grocery merchants to calibrate their inventory levels by assessing buyer purchasing behavior and requested products from various viewpoints and situations \cite{12}. \\

For example, grocery retailers might need to screen cycles like when customers go for particular nourishment, purchasing patterns amid sales deals when storing activity peaks or seasonally inspired buys. As indicated by a report from Manthan, this methodology worked for U.K. food grocery merchant Waitrose: a deeper understanding of buyer purchasing behavior and demand outlines using cutting edge client analytics and predicting tools helped the store \cite{13}. Concurrently, retailers can utilize these systems to all the more deftly change their stock levels and amplify high-buy products. 


\subsection{Leveraging Predictive Analytics}

Amazon spearheaded item proposal engine: the "if you purchased that, you may like this" invention. This strategic changing web-based shopping feature mirrors the retailer's profound assessment of buyers' shopping basket.  Proposal engine is intended to enable customers to find items they were not sorting out but rather would be interested in purchasing \cite{10}. Today, general grocery merchants are progressively tapping the global innovation behind proposal engine: predictive analytics. This kind of assessment measures future patterns in light of present and past information, and it can enable stores to improve business. Information is driven, all-encompassing assessment of "purchasing triggers, for example, regularity, weather, stock, and advancements, is progressively informing grocery stores' product blend, marketing plans, and sales forecast \cite{14}. Furnished with these information-driven tools, stores can better distinguish what items customers need today and what they will be demanding in future, and this learning will enable them to stay competitive for a considerable length of time to come.  

\section{Introduction}

 Digitization set apart by an increasing number social media and mobile devices is shifting the business landscape in every sector insurance included. The opportunity presented by this aspect for insurance companies are immense. Communities and social networks enable insurers to interface with their clients better, which to their advantage improves branding, customer retention, and acquisition \cite{12}. Insurance companies additionally get a plenty of contributions from computerized data as feedbacks, which likewise can be utilized to develop unique products and aggressive valuing. Digitization of big data analytics offers numerous opportunities that Insurances Company can harness to detect fraud among their customers. Dealing with fraud manually has dependably been expensive for insurance firms regardless of the possibility that maybe a couple of minor fraud went undetected \cite{16}. What's more, the trends in big data (the evolution in unstructured information) are prone to numerous fraud, which can go without notice is analysis is performed correctly. In the proceeding section, the article will examine important of big data in insurance fraud detection and its relevancy.  

\section{IMPORTANCE BIG DATA AND INSURANCE FRAUD DETECTION}
Conventionally, insurance firms utilize statistical models to recognize fraudulent cases. These models have their limitation \cite{18}. To start with, they employ sampling techniques to assess information, which prompts at least one fraud going unnoticed. There is a punishment for not performing a proper assessment of the data provided. Subsequently, this strategy depends on the cases analyzed before. Therefore, every time different fraud takes place, insurance firms need to manage the impact for the first time. Lastly, the conventional strategy works in silos and is not correctly equipped for taking care of the natural developing wellsprings of data from various diverts and diverse capacities in an integrated way. Analytics tends to be difficult and assumes an exceptionally pivotal part in fraudulent recognition for insurance firms. In the proceeding section, the significant benefits of utilizing big analytics in fraud detection assessed.

\subsection{Identification of low incidence events: }
Utilizing sampling methods accompanies its particular arrangement of acknowledged mistakes. By using analytics, insurance can manufacture frameworks that go through every fundamental datum. This like this distinguishes events with low frequency (0.001{\%}) \cite{17}. Methods such as predictive modeling can be utilized to altogether break down processes of fraud, channel clear cases, and allude low-rate fraud cases for facilitating analytics. 



\subsection{Enterprise-wide solution:}

Analytics help in building a global point of view of the anti-fraud endeavors all through the undertaking. Such a point of view regularly prompts dominant fraud location by connecting related data inside the association. Fraud can happen at various source focuses premium, claims or surrender, application, employee-related or outsider fraud. In the meantime, insurance channel broadening is adding to the breakdown of identifiable information. Insurance-related exercises should be possible using cell phones separated from the conventional face-to-face and online Insurance \cite{18,1}. This can be seen as an expansion to data storehouses in the Insurance business. Given more prominent channel enhancement and the development of ranges where fraud can happen, it is vital for insurers to have reachable enterprise-level data about their business and clients. 

\subsection{Data Integration:}

Analytics assumes a vital part in incorporating information. Viable fraud recognition abilities can be worked by joining information from different sources. Analytics additionally help in integrating inside information with outsider information that may have predictive significance, for example, public records. Information sources with derogatory properties are on the whole public documents that can be incorporated into a model. Cases include liquidations, liens, criminal records, judgment, abandonment, or even deliver change speed to show transient conduct. Different sorts of outsider information can be useful in upgrading effectiveness, for example, audit evaluating data to decide whether harms coordinate portrayal or misfortune or injury being guaranteed \cite{16}. A standout amongst the most under-used information sources is doctor's visit expense audit information. This information, if utilized as a part of a model legitimately, is a gold dig for organizations researching medical fraud. Revealing peculiarities, in charging and adding these to the next scoring motors or interpersonal organization analytics will diminish the measure of time an agent or expert spends endeavoring to pull the majority of the pieces together to recognize deceitful action. 

\subsection{Harnessing Unstructured Data:}

Analytics is useful for getting the best incentive from unstructured information. Fraud can be delicate or hard. This depends on whether it comprises of a policyholder's misrepresented cases, or on the off chance that it contains of a policyholder arranging or creating a misfortune. At an abnormal state, fraud can happen amid commission discounting, because of false documentation, an arrangement between parties or from is offering \cite{12}. Albeit bunches of organized data is put away in an information distribution center as a component of numerous applications, a significant portion of the vital data about a fraud is in unstructured information, for example, outsider reports, which are not assessed. In most insurance firms, data accessible in online networking is not suitably stored. An uncommon investigative-unit specialist will concur that unstructured information is vital for fraud examination. Since textual information is not straightforwardly utilized for reporting, it does not discover a place in most information stockrooms \cite{17}. This is the place content examination can assume a crucial part in checking on this unstructured information and giving some valuable experiences in fraud discovery.


\section{RELEVANCE OF BIG DATA IN INSURANCE FRAUD DETECTION}
Big data analytics is a reality for the insurance company because of its capability to enhance various conventional technologies and be used to detect fraudulent acts. In the proceeding section, the relevance of big data and insurance fraud detection will be examined.

\subsection{Text analysis}

In numerous Insurance fraud recognition ventures, from 33{\%} to one-portion of factors utilized as a part of the fraud location model originate from unstructured content data. This is particularly helpful for long-tail claims, for example, damage claims, because the best information frequently is found in claim notes \cite{18} . Content mining is something beyond keyword sorting. Excellent content analytics apparatuses translate the importance of the words to establish context. Innovation that is adroit at preparing common dialect can help remove factors from the unstructured content that can be utilized for assist fraud modeling.



\subsection{Data Management}

Regardless of where your information is stored — from legacy frameworks to the valid information stockpiling structure, Hadoop — an information administration framework can enable insurers to make reusable information rules. They give a standard, repeatable strategy for enhancing and incorporating information \cite{17}. Preferably, you need a framework that interfaces with different information sources. It ought to have streamlined information league, relocation, synchronization, organization, and visual assessment. 


\subsection{Event Stream Processing}

This enables insurers to investigate and processes in movement (i.e., process streams). Rather than putting away information and running questions against data, you store the inquiries and stream the data through them \cite{12}. This is foundational to both ongoing fraud identification (invigorating fraud scoring) and successful utilization of great high-speed information sources similar to vehicle telematics. 

\subsection{Hadoop}
A free programming structure that assesses and prepares of tremendous collected information in a distributed environment of computing. It offers gigantic details stockpiling and super-quick processing at around 5 percent of the cost of convection less-adaptable databases. Hadoop's mark quality is the capacity to deal with organized and unstructured information (counting sound, text, and visual), and in expansive volumes. Insurers either can employ Hadoop specialists to exploit the structure or purchase items that scaffold to existing databases and information distribution centers\cite{16,17}. This foundational innovation for making predictive analytics models stays one-step in front of fraudsters and spillage of paid-out cases cash. The exchange observing advancement innovation used to battle regularly complicated illegal tax avoidance utilizes Hadoop as a center stockpiling and sorting out innovation. Complex organized crack rings and therapeutic factories, for instance, are conveying progressively modern techniques for laundering cash stolen from auto insurers.

\subsection{In memory}
In-memory analytics is a processing style in which all information utilized by an application is put away inside the principal memory of the computing condition. Instead of being available on a disc, the data stays suspended in the mind of useful sets of PCs. Different clients can share this information with numerous applications in a quick, secure, and simultaneous way. In-memory analytics likewise exploits multi-threading and distributed registry \cite{12,16}. This implies clients can disseminate the information (and complex workloads that process the data) over different machines in a group or inside a single server condition. In-memory analytics manages questions and information analytics, yet also is utilized with more-complex procedures, for example, predictive analytics, machine learning, and analytics. The sorts of neural-network analytics that assist insurer in discovering association among suspects sustaining claim and premium fraud depending on the kind of processes

\subsection{Software as a Service (SaaS)}
Predictive modeling and different analytics were accessible to large insurance net providers willing to introduce the innovation on location as of not long ago. Software as a service has advanced to even where genuinely little insurers can exploit Big Data analytics \cite{16}. Insurance providers subscribe to a service keeps running by a seller as opposed to paying for the vast buy, establishment, and support of in-house frameworks. SaaS likewise is named "on-demand software."


\section{DISCUSSION}
Form such progress we can see the diverse reach that big data has and how it affects the users in their experiences with big data. Not only do we see big data creating advertising products to consumers, we are also seeing social media sites using big data to influence other functions of consumer’s daily lives. To further that observation, there is necessity behind seeing the claims that Facebook makes on its ability to influence its consumers to influence those within the social media sphere. If big data has the access to arrange itself around the sphere of the consumer to have the consumer act on certain task, what does that mean the uses of big data are to social media sites? We can assume from our knowledge that social media sites like Facebook could be interjecting into the private lives of their users by instigating on the data that is collected in processed to pinpoint user habits. It can also be seen that big data facilitates the necessary components of information to allow social media sites to specify their own approaches to their consumer in ways that can be seen as going over the line when it comes to their connection with their consumers. It is however, still a very enterprise avenue of using big data. As it allows for the social media to influence social, economical, and political landscapes. But that in itself is also a very dangerous power to have. As those who use the big data and direct their resources into specific marketing strategies can alter nearly whatever they’d like to in front of the irrational consumer. \\

As we have observed of big data above, we also learn of the prediction value and how big data escalates the ability for businesses to predict and recommend to consumers different products. The use of pooling data and sifting it through algorithms in order to precisely choose methods of spawning products before consumers is a fundamental use of big data. Big data becomes a tool that assimilates the data that is created to businesses to create more big data. The process is unending and constantly provides businesses with unlimited amounts of data that can be used to a spearhead their campaigns. Does big data then become a commodity that is used like currency to businesses? Well, it is very possible. As the use of big data is how businesses maneuver their strategies to get consumer to consume. If these algorithms meant to increase sales were used for something else, say medical awareness to the issues that exists within smoking cigarettes, how will the big data be used and what forms of algorithms would be used? The use of prediction analyst suits sales, but based on the observation made above, it is probably even more effective in helping to create a knowledgeable public. By facilitating the big data and being able to sort out the necessary public images that have control over social sphere through mass social medias, there can be an exchange of data between consumers. Because of the interplay between data and how consumers absorb them and create more data that is spawned for more information, big data can in turn control knowledgeable outcomes in public opinions. The use of big data is vast then when it comes to understanding the many components that make up what big data is.\\ 

Our observation above also places the consumer experience as an important facilitator for big data to exists. The habits and practices of consumers as well as the opinions and locations of the consumer can truly inhibit how big data is filtered to facilitate the necessary components of data to market and process information. This process can of course be seen using AI technology in order to expedite the data that is coming in. It does this so that the data is filtered and able to be used to immediately influence commercial markets, social media spheres, and consumer habits. That in turn regulates and begins to push out even more big data from the interactions that consumer have with the new platforms that are created from old big data that was used in order to create their new purchases or opinions. AI technology then becomes a fundamental component to the access, collection, analysis, and interpretation of big data. Its use of manipulating and translating data in order to be used to create enterprise is crucial to the development of more big data. From the observation above, it can also be said that AI technology will has also positioned itself in a way where it has become fundamental to the big data analysis and because of that, AI technology is part of big data.\\ 

The paper also sought to observe the nature of consumers having the capacity to control the big data that flows into collection. The use comes from filter settings likes those included on Amazon in order to help narrow searches of items or on Netflix in order to create the right kind of streaming that the consumer requires or likes. If that is the case, the consumer actually holds a lot of power when it comes to the collection, analysis, and interpretation process of big data. Within how the consumer chooses to reside over these social medias and commercial businesses determines how the social media sites and businesses get their data. Beyond that, the consumer has no knowledge of how to analyze or interpret big data, yet holds the key to the very idea of big data. Because of this notion, the discussion here seeks to try and highlight the importance of businesses maintaining and using data collected responsibly. \\

Big data is used in order to interact with consumers in order to sell or sway. The use of data however is also created by consumers. For this symbiotic relationship to exists and stay peaceful, businesses must be sure that they are not over stepping privacy issues when it comes to the use of big data. By enterprising methods to help consumers with choices and options through their recommendation systems and early predictive measures of trend by their collected big data, that is fine. But when business pressure consumers with the use of big data, the business will most likely end up losing new data to collect. As if no one is using their sources to create data, their lack of data causes them to have slow flow, and that leads to isolation of data.\\ 

Take for instance the process of data mining. Data mining is used in order to received particular forms of information about consumers. This information is in the form of data, this data is put through special filters to narrow in on what it is that businesses want to know about particular groups of consumers to achieve the best methods of interacting with the consumers in order to highlight necessary products to the consumer. What happens if the algorithms for this particular data mine was off? This would mean that the data that was supposed to continue in the line of procession ends up lost. Because missing the mark with data mining and interpretation means that businesses loses their edge with their consumers. \\

Big data is a very complex topic to talk about. It is however, a very interesting topic to look at. As when we are observing what forms of big data are used in order to create experiences for consumers and business practices for businesses we can see the importance of having a very strong handle on the idea of big data. It is not just a process by which you collect massive amounts of information and then through it back out into a market. Big data must be molded around using algorithms, AI technology, studies done to mine particular forms of data, and even understanding the complex notions of unstructured data. Because of these reasons, the study of big data is still relatively incomplete. The use of big data however, should be understood as a relationship between consumers and those who seek to use big data to facilitate their individual means. 

\section{CONCLUSION}
The paper sought out to examine the complexities of big data, but to be more precise, this paper seeks out to the multiplicity of avenues in which big data is used by our technological world in regards to Online shopping, Streaming Services, and Social Medias. The conclusion is that the multiple complex systems which make up the forefront and system of enterprise around big data falls under the very distinctive relationship that exists around the users of these services, and the sources the users use in order to create more big data. \\ 

Such complex multiplicity of diversified uses alter the understanding of big data by showcasing that big data in itself is easily manipulated and altered. This becomes the case because of the multiple layers of data that exists in any given moment. The use of these data are incorporated in a way that there are many organization that are still trying to spearhead further in the endeavors of big data. The papers observation of the multiple forms of big data conversions, analytics, prediction standards, and even experimental uses of data reinforces the concept that big data is as it is called, massive. Because of such presence, to truly be able to look into the multiplicity of big data would mean a massive overhaul of research meant to showcase the existence. This paper however does not do that, but would also rather seek to showcase that.\\ 

Online shopping and streaming sites uses a multiplicity of tools alongside big data in order to function with consumers and creates a diversified experience for consumers. The tools that are used by these shopping and steaming businesses alter big data into sustainable forms of information that are then used in order to predict and recommend to consumers what products to purchase and recommendations. These are all done through algorithms used to analyze the data. The paper observes and concludes that the standing made by these businesses are diversified and are meant to showcase the suitable substance of the big data to consumers. The use their procedures not only diversify the consumer experience but also diversifies the way that big data is collected and used. Big data within these forms of principles are collected and retained under algorithmic databases that are then filtered out when it is being generated by consumers. The process of big data filtering is not only done by businesses, they are also given as options to consumers. Through the process of filtering consumers can do the exact same thing. \\

Social media sites use big data just at online shopping and streaming services do, but social media also has another power. One which allows them to facilitate their studies into experiences around the consumer. By doing this, social media can use big data in order to influence and manipulate consumers into specific acts of studies that are being done by the social media sites. This form of big data usage not only diversifies but also includes possibility for growth in collection of information. As when social media analysis these complex forms of data known as unstructured data, they are having a deeper perspective into consumer habits, wants, and additional personal information that expands their usage of big data. \\

Big data is a very diversified entity. Even though it can be narrowed down to certain institutions or entities, it in itself is able to expand largely in those narrowed views. The diversification of big data is very crucial to the survival and usage of big data. Without multiple sources to collect necessary data, there would be no big data. Therefore, the user’s experiences around big data needs to be one that is flexible and seeks to incorporate the right amounts of AI and Algorithms in order to maintain steady flow of data which encapsulates the very idea of diversified big data.\\ 

In summary, the multiplicity of avenues that exists around big data creates the very core study that it takes in order to understand the practices and exhibits which are induced by the use of big data itself. By observing real world applications of big data we can see that the diversification of big data does not need to be mellowed or shallowed out from the perspective of a programmer, as it seems that the market capitalizes on big data and therefore creates the very enterprise of multiplicity within its diversification. 

\begin{acks}

My group work very hard to facilitate a study around diversification and observation. Their hard work in reading through these sources multiple times to see the rights of the observation is very much appreciated. We would also like to thank our professor for the chance to take on such a free ranging topic, as it has allowed us to further appreciate big data and its importance in our future endeavors.  

\end{acks}

\bibliographystyle{ACM-Reference-Format}
\bibliography{report} 

\end{document}